\maketitle
\makedeclaration

\begin{abstract} % 300 word limit
In Click-Through Rate (CTR) estimation problems for online advertising, besides the pursue of fancy models and algorithms, feature engineering is informal but absolutely vital to its success. Traditionally, in industry, the generation, combination and transformation of effective one-hot encoded features are always conducted manually, which is time-consuming and labor-intensive. Even worse, in the age of big data, astronomical user impressions lead to enormous unique binary features, the industry is having an urgent demand for the type of ad feature which are light-weight, scalable and auto-generated. In this paper, we propose the concept of \textit{counting features} which constitutes the \textit{Frequency} and \textit{Average CTR} information of all the items in each field, representing the statistical characteristics of the dataset. Further, we exploit the performance of counting feature using valuable real world data provided by Adform. AUC and RMSE are used to evaluate its performance in CTR prediction showing that the its CTR prediction accuracy when utilising non-linear model is comparable to that of binary feature utilising linear model. Finally, we apply counting feature on \textit{cross-domain learning} problem, with the goal to solve the canonical \textit{cold start} problem for online adverting CTR prediction, so that knowledge from old ad campaigns can be directly used by new campaign with minimal updates for the classifiers. We show that the distributions of the conditional probability \(Pr(Click|Feature)\) are alike among different campaigns, the variability of classifiers can be decreased when using counting feature against binary feature, thus increasing the generality of CTR prediction model and making \textit{direct cross domain CTR prediction model} possible. Experiment results show counting feature's performance is superior to binary feature in cold start problem. 
\end{abstract}

\begin{acknowledgements}
I would like to pay my best respect to my supervisor Dr. Jun Wang who provides invaluable guidance and support to my research, only with the ongoing creative and solid suggestions from Dr. Jun can I finalize my Msc project.

I also want to tribute to the PHD candidate, Weinan Zhang, who provides precious help for my mathematical deduction work and experiment implementation, as the final year PHD who is up to his neck, Weinan still takes a lot of time off in supporting me, I highly appreciate his guidance.

The senior fellow student Xuyang Wu also provides me with many helps in setting up the system of \textit{OpenBidder}\footnote{http://www.openbidder.com/}, I learned a lot of engineering work for RTB system from him, I would like to thank him.

Last but not least, I would absolutely thank 	
Mr.Thomas Furmston and Mr.Edward Snelson from Adform company who not only provide me with valuable real world advertising data, but also enormous guidance for my project, I will show my best respects to them.
\end{acknowledgements}

\setcounter{tocdepth}{2} 
% Setting this higher means you get contents entries for
%  more minor section headers.

\tableofcontents
\listoffigures
\listoftables

